\documentclass[12pt,a4paper]{article}

% Paquetes
\usepackage[utf8]{inputenc}
\usepackage[spanish]{babel}
\usepackage{geometry}
\geometry{left=2.5cm,right=2.5cm,top=3cm,bottom=3cm}
\usepackage{graphicx}
\usepackage{amsmath}
\usepackage{amssymb}
\usepackage{float}
\usepackage{listings}
\usepackage{xcolor}
\usepackage{hyperref}
\usepackage{fancyhdr}
\usepackage{tcolorbox}
\usepackage{enumitem}
\usepackage{booktabs}
\usepackage{longtable}

% Configuración de listings para Python
\lstdefinestyle{pythonstyle}{
    language=Python,
    basicstyle=\ttfamily\small,
    keywordstyle=\color{blue}\bfseries,
    commentstyle=\color{gray}\itshape,
    stringstyle=\color{red},
    numbers=left,
    numberstyle=\tiny\color{gray},
    stepnumber=1,
    numbersep=8pt,
    backgroundcolor=\color{gray!10},
    showspaces=false,
    showstringspaces=false,
    showtabs=false,
    frame=single,
    tabsize=4,
    captionpos=b,
    breaklines=true,
    breakatwhitespace=false
}

\lstdefinestyle{bashstyle}{
    basicstyle=\ttfamily\small,
    keywordstyle=\color{blue}\bfseries,
    commentstyle=\color{gray}\itshape,
    backgroundcolor=\color{gray!10},
    frame=single,
    breaklines=true
}

% Configuración de tcolorbox
\tcbuselibrary{skins,breakable}

\newtcolorbox{infobox}[1]{
    colback=blue!5!white,
    colframe=blue!75!black,
    fonttitle=\bfseries,
    title=#1,
    breakable
}

\newtcolorbox{warningbox}[1]{
    colback=orange!5!white,
    colframe=orange!75!black,
    fonttitle=\bfseries,
    title=#1,
    breakable
}

\newtcolorbox{successbox}[1]{
    colback=green!5!white,
    colframe=green!75!black,
    fonttitle=\bfseries,
    title=#1,
    breakable
}

% Configuración de headers
\pagestyle{fancy}
\fancyhf{}
\fancyhead[L]{Tutorial del Sistema de Modelado de Esterificación}
\fancyhead[R]{\thepage}
\renewcommand{\headrulewidth}{0.4pt}

% Información del documento
\title{
    \vspace{-2cm}
    \textbf{Tutorial Paso a Paso}\\
    \Large{Sistema Integrado de Modelado de Esterificación\\
    para Producción de Biodiésel}\\
    \vspace{1cm}
    \includegraphics[width=0.3\textwidth]{logo.png}\\
    \vspace{1cm}
}
\author{Sistema de Modelado Cinético\\
        Versión 1.0}
\date{\today}

\begin{document}

\maketitle
\thispagestyle{empty}

\newpage
\tableofcontents
\newpage

%==============================================================================
\section{Introducción}
%==============================================================================

\subsection{Propósito de este Tutorial}

Este tutorial proporciona una guía completa paso a paso para utilizar el \textbf{Sistema Integrado de Modelado de Esterificación} para la producción de biodiésel mediante transesterificación catalizada por CaO. Está diseñado para que usuarios sin experiencia previa puedan:

\begin{itemize}
    \item Preparar sus datos experimentales en el formato correcto
    \item Ejecutar todos los módulos del sistema
    \item Interpretar los resultados obtenidos
    \item Generar reportes y gráficas publication-ready
\end{itemize}

\subsection{Características del Sistema}

El sistema incluye \textbf{13 módulos} (más de 6500 líneas de código) que cubren:

\begin{enumerate}[label=\arabic*)]
    \item \textbf{Procesamiento de datos GC-FID}: Conversión de cromatogramas a concentraciones
    \item \textbf{Modelos cinéticos}: 1 paso (simplificado) o 3 pasos (mecanístico)
    \item \textbf{Ajuste de parámetros}: Estimación de energía de activación (Ea) y factor pre-exponencial (A)
    \item \textbf{Optimización}: Búsqueda de condiciones óptimas (T, RPM, \% catalizador)
    \item \textbf{Análisis comparativo}: Métricas estadísticas entre modelos
    \item \textbf{Visualización}: Gráficas profesionales y reportes automatizados
    \item \textbf{Especificaciones CFD}: Diseño completo para reactor 20L (hélice axial, sin baffles, con serpentín)
\end{enumerate}

\subsection{Requisitos del Sistema}

\begin{infobox}{Requisitos Mínimos}
\textbf{Software requerido}:
\begin{itemize}
    \item Python 3.8 o superior
    \item pip (gestor de paquetes)
    \item Editor de texto (VS Code, PyCharm, Spyder, etc.)
\end{itemize}

\textbf{Software opcional}:
\begin{itemize}
    \item Ansys Fluent (para simulación CFD)
    \item LaTeX (para compilar documentación)
\end{itemize}

\textbf{Hardware recomendado}:
\begin{itemize}
    \item RAM: 4 GB mínimo, 8 GB recomendado
    \item Almacenamiento: 500 MB libres
    \item Procesador: Dual-core 2.0 GHz o superior
\end{itemize}
\end{infobox}

\subsection{Estructura del Tutorial}

El tutorial sigue un flujo lógico de trabajo:

\begin{enumerate}
    \item \textbf{Instalación y configuración inicial} (Sección 2)
    \item \textbf{Preparación de archivos de datos} (Sección 3)
    \item \textbf{Ejecución de módulos principales} (Sección 4)
    \item \textbf{Interpretación de resultados} (Sección 5)
    \item \textbf{Casos de uso avanzados} (Sección 6)
    \item \textbf{Troubleshooting} (Sección 7)
\end{enumerate}

%==============================================================================
\section{Instalación y Configuración Inicial}
%==============================================================================

\subsection{Paso 1: Clonar el Repositorio}

\begin{enumerate}[label=\textbf{\arabic*.}]
    \item Abrir una terminal (Linux/Mac) o Command Prompt (Windows)
    \item Navegar al directorio donde deseas instalar el sistema
    \item Ejecutar:
\end{enumerate}

\begin{lstlisting}[style=bashstyle]
git clone <repository-url>
cd mod_esterificacion
\end{lstlisting}

\begin{warningbox}{¿No tienes Git?}
Si no tienes Git instalado:
\begin{itemize}
    \item \textbf{Windows}: Descargar de \url{https://git-scm.com/download/win}
    \item \textbf{Linux}: \texttt{sudo apt install git} (Ubuntu/Debian)
    \item \textbf{Mac}: Viene preinstalado o usar Homebrew: \texttt{brew install git}
\end{itemize}

Alternativamente, descarga el repositorio como ZIP desde GitHub y descomprímelo.
\end{warningbox}

\subsection{Paso 2: Crear Entorno Virtual (Recomendado)}

Los entornos virtuales aíslan las dependencias del proyecto.

\subsubsection{En Linux/Mac}

\begin{lstlisting}[style=bashstyle]
python3 -m venv venv
source venv/bin/activate
\end{lstlisting}

\subsubsection{En Windows}

\begin{lstlisting}[style=bashstyle]
python -m venv venv
venv\Scripts\activate
\end{lstlisting}

\begin{infobox}{Verificación}
Si el entorno virtual está activado, verás \texttt{(venv)} al inicio de la línea de comandos:
\begin{verbatim}
(venv) user@computer:~/mod_esterificacion$
\end{verbatim}
\end{infobox}

\subsection{Paso 3: Instalar Dependencias}

Con el entorno virtual activado:

\begin{lstlisting}[style=bashstyle]
pip install --upgrade pip
pip install -r requirements.txt
\end{lstlisting}

Este proceso instalará todas las bibliotecas necesarias:

\begin{table}[H]
\centering
\caption{Principales dependencias instaladas}
\begin{tabular}{ll}
\toprule
\textbf{Paquete} & \textbf{Propósito} \\
\midrule
numpy & Cálculos numéricos \\
scipy & Integración de ODEs y optimización \\
pandas & Manipulación de datos \\
matplotlib & Visualización 2D \\
plotly & Visualización interactiva 3D \\
lmfit & Ajuste de parámetros no lineales \\
openpyxl & Exportación a Excel \\
\bottomrule
\end{tabular}
\end{table}

\subsection{Paso 4: Verificar Instalación}

Ejecutar el siguiente comando para verificar que todos los módulos se importan correctamente:

\begin{lstlisting}[style=pythonstyle]
python -c "from src.models.kinetic_model import KineticModel; print('OK')"
\end{lstlisting}

\begin{successbox}{Instalación Exitosa}
Si ves \texttt{OK} sin errores, la instalación fue exitosa. Estás listo para continuar.
\end{successbox}

%==============================================================================
\section{Preparación de Archivos de Datos}
%==============================================================================

\subsection{Introducción}

El sistema requiere que tus datos experimentales estén en formatos específicos. Esta sección explica cómo preparar:

\begin{itemize}
    \item Datos de cromatografía GC-FID
    \item Variables experimentales
    \item Archivos de configuración
\end{itemize}

\textbf{Ubicación de plantillas}: La carpeta \texttt{plantillas/} contiene archivos de ejemplo listos para llenar.

\subsection{Opción 1: Datos GC-FID en CSV}

\subsubsection{Formato del Archivo}

Los datos GC-FID deben estar en formato CSV (valores separados por comas) con las siguientes columnas:

\begin{table}[H]
\centering
\caption{Columnas requeridas en CSV de datos GC}
\begin{tabular}{lll}
\toprule
\textbf{Columna} & \textbf{Descripción} & \textbf{Unidades} \\
\midrule
tiempo\_min & Tiempo de muestreo & minutos \\
compuesto & Nombre del compuesto & texto \\
area\_pico & Área del pico cromatográfico & unidades arbitrarias \\
tiempo\_retencion\_min & Tiempo de retención & minutos \\
notas & Notas adicionales (opcional) & texto \\
\bottomrule
\end{tabular}
\end{table}

\textbf{Compuestos válidos}:
\begin{itemize}
    \item \texttt{TG}: Triglicéridos
    \item \texttt{DG}: Diglicéridos (opcional)
    \item \texttt{MG}: Monoglicéridos (opcional)
    \item \texttt{FAME}: Ésteres metílicos de ácidos grasos
    \item \texttt{GL}: Glicerol
    \item \texttt{MeOH}: Metanol
    \item \texttt{Estándar\_Interno}: Estándar interno para cuantificación
\end{itemize}

\subsubsection{Ejemplo de Archivo}

\begin{lstlisting}[style=bashstyle, caption=Ejemplo: experimento\_01.csv]
tiempo_min,compuesto,area_pico,tiempo_retencion_min,notas
0,TG,15000.0,12.5,Tiempo inicial
0,MeOH,50000.0,2.1,Tiempo inicial
0,FAME,0.0,8.3,Tiempo inicial
0,GL,0.0,14.2,Tiempo inicial
0,Estándar_Interno,25000.0,10.0,Tiempo inicial
10,TG,13500.0,12.5,Muestra a 10 minutos
10,MeOH,48000.0,2.1,Muestra a 10 minutos
10,FAME,1500.0,8.3,Muestra a 10 minutos
10,GL,150.0,14.2,Muestra a 10 minutos
10,Estándar_Interno,25000.0,10.0,Muestra a 10 minutos
...
\end{lstlisting}

\subsubsection{Pasos para Crear tu Archivo}

\begin{enumerate}
    \item Copiar la plantilla:
    \begin{lstlisting}[style=bashstyle]
cp plantillas/plantilla_datos_gc.csv data/raw/mi_experimento.csv
    \end{lstlisting}

    \item Abrir \texttt{data/raw/mi\_experimento.csv} en Excel, LibreOffice Calc o editor de texto

    \item Reemplazar los valores de ejemplo con tus datos reales

    \item \textbf{Importante}: Mantener el formato CSV y los nombres de columnas exactos

    \item Guardar el archivo (asegurarse que sigue siendo .csv, no .xlsx)
\end{enumerate}

\begin{warningbox}{Errores Comunes}
\begin{itemize}
    \item \textbf{Separador incorrecto}: Usar coma (,) no punto y coma (;)
    \item \textbf{Nombres de compuestos}: Respetar mayúsculas/minúsculas exactas
    \item \textbf{Formato decimal}: Usar punto (12.5) no coma (12,5)
    \item \textbf{Estándar interno ausente}: Debe aparecer en TODOS los tiempos de muestreo
\end{itemize}
\end{warningbox}

\subsection{Opción 2: Experimento Completo en JSON}

\subsubsection{Ventajas del Formato JSON}

El formato JSON permite especificar todas las variables del experimento en un solo archivo:

\begin{itemize}
    \item Variables del experimento (temperatura, RPM, etc.)
    \item Propiedades del aceite y metanol
    \item Características del catalizador
    \item Condiciones de operación
    \item Datos GC-FID integrados
\end{itemize}

\subsubsection{Estructura del Archivo JSON}

Ver archivo \texttt{plantillas/plantilla\_experimento.json}. Las secciones principales son:

\begin{enumerate}
    \item \textbf{experimento}: Metadatos (ID, fecha, operador)
    \item \textbf{reactivos}: Aceite, metanol, relación molar
    \item \textbf{catalizador}: Tipo (CaO), masa, pureza
    \item \textbf{condiciones\_reaccion}: T, presión, RPM, tiempo
    \item \textbf{muestreo}: Tiempos y volúmenes
    \item \textbf{analitica}: Configuración GC-FID
    \item \textbf{datos\_gc}: Referencia a archivo CSV o datos inline
\end{enumerate}

\subsubsection{Pasos para Usar la Plantilla JSON}

\begin{enumerate}
    \item Copiar la plantilla:
    \begin{lstlisting}[style=bashstyle]
cp plantillas/plantilla_experimento.json mi_experimento_01.json
    \end{lstlisting}

    \item Editar con un editor de texto (VS Code, Sublime, Notepad++)

    \item Llenar todos los campos con tus datos reales

    \item \textbf{Validar JSON}: Usar validador online (ej. \url{https://jsonlint.com})

    \item Guardar el archivo
\end{enumerate}

\begin{infobox}{Tip: Múltiples Experimentos}
Para múltiples experimentos a diferentes temperaturas:
\begin{itemize}
    \item Crear \texttt{experimento\_55C.json}
    \item Crear \texttt{experimento\_65C.json}
    \item Crear \texttt{experimento\_75C.json}
    \item Modificar solo los campos que cambian (temperatura, datos GC)
\end{itemize}
\end{infobox}

\subsection{Opción 3: Archivo de Configuración YAML (Avanzado)}

El archivo \texttt{config.yaml} permite configurar parámetros globales del sistema.

\subsubsection{Pasos para Configurar}

\begin{enumerate}
    \item Copiar plantilla:
    \begin{lstlisting}[style=bashstyle]
cp plantillas/plantilla_config.yaml config.yaml
    \end{lstlisting}

    \item Editar secciones relevantes:
    \begin{itemize}
        \item \texttt{kinetic\_model}: Tipo de modelo, parámetros iniciales
        \item \texttt{optimization}: Variables a optimizar, algoritmo
        \item \texttt{visualization}: Formatos de gráficas, estilo
    \end{itemize}

    \item Guardar en la raíz del proyecto
\end{enumerate}

\begin{warningbox}{Configuración Opcional}
El sistema funciona sin \texttt{config.yaml} usando valores por defecto. Solo créalo si necesitas personalizar parámetros avanzados.
\end{warningbox}

%==============================================================================
\section{Ejecución de Módulos Principales}
%==============================================================================

\subsection{Introducción al Flujo de Trabajo}

El sistema tiene 5 modos de operación principales, ejecutables vía:

\begin{enumerate}
    \item \textbf{Línea de comandos} (CLI): \texttt{python main.py --mode ...}
    \item \textbf{Scripts de ejemplo}: \texttt{python ejemplo\_XX\_nombre.py}
    \item \textbf{Programático}: Importar módulos en tus propios scripts
\end{enumerate}

Esta sección cubre el flujo típico de trabajo.

\subsection{Módulo 1: Procesamiento de Datos GC-FID}

\subsubsection{Objetivo}

Convertir áreas de picos cromatográficos a concentraciones molares y calcular conversiones.

\subsubsection{Método 1: Línea de Comandos}

\begin{lstlisting}[style=bashstyle]
python main.py --mode process_gc \
    --input data/raw/experimento_01.csv \
    --output data/processed/
\end{lstlisting}

\textbf{Parámetros}:
\begin{itemize}
    \item \texttt{--mode}: \texttt{process\_gc}
    \item \texttt{--input}: Ruta al archivo CSV con datos GC
    \item \texttt{--output}: Directorio donde guardar resultados procesados
\end{itemize}

\subsubsection{Método 2: Script de Ejemplo}

\begin{lstlisting}[style=bashstyle]
cp plantillas/ejemplo_01_procesar_gc.py .
python ejemplo_01_procesar_gc.py
\end{lstlisting}

\textbf{Antes de ejecutar}, editar el script para ajustar:
\begin{itemize}
    \item \texttt{INPUT\_FILE}: Ruta a tu archivo CSV
    \item \texttt{C\_TG\_INICIAL}: Concentración inicial de TG (mol/L)
    \item \texttt{TEMPERATURA}: Temperatura del experimento (°C)
\end{itemize}

\subsubsection{Resultados Generados}

\begin{enumerate}
    \item \textbf{Archivo CSV procesado}: \texttt{data/processed/resultados\_gc\_procesados.csv}

    Contiene:
    \begin{itemize}
        \item Tiempo (min)
        \item Concentraciones: C\_TG, C\_MeOH, C\_FAME, C\_GL (mol/L)
        \item Conversión (\%)
        \item Rendimiento FAME (\%)
    \end{itemize}

    \item \textbf{Gráficas}: \texttt{results/figures/procesamiento\_gc.png}

    Incluye 4 subgráficas:
    \begin{itemize}
        \item Conversión vs tiempo
        \item Perfiles de concentración (TG, FAME, GL)
        \item Rendimiento de FAME
        \item Balance de reactivos (TG, MeOH)
    \end{itemize}

    \item \textbf{Salida en consola}: Estadísticas del procesamiento
\end{enumerate}

\subsubsection{Interpretación de Resultados}

\begin{table}[H]
\centering
\caption{Métricas clave del procesamiento GC}
\begin{tabular}{lll}
\toprule
\textbf{Métrica} & \textbf{Significado} & \textbf{Valores típicos} \\
\midrule
Conversión final & \% de TG convertido & 85-95\% (óptimo) \\
Rendimiento FAME & \% de FAME producido & 80-90\% \\
Selectividad & FAME/productos totales & > 90\% \\
\bottomrule
\end{tabular}
\end{table}

\begin{successbox}{¿Qué Sigue?}
Una vez procesados los datos GC, el siguiente paso es ajustar los parámetros cinéticos (Sección 4.3).
\end{successbox}

\newpage
\subsection{Módulo 2: Ajuste de Parámetros Cinéticos}

\subsubsection{Objetivo}

Determinar los parámetros cinéticos ($A$ y $E_a$) que mejor ajustan los datos experimentales.

\textbf{Ecuación de Arrhenius}:
\begin{equation}
k(T) = A \cdot \exp\left(-\frac{E_a}{RT}\right)
\end{equation}

donde:
\begin{itemize}
    \item $k(T)$: Constante de velocidad (min$^{-1}$)
    \item $A$: Factor pre-exponencial (min$^{-1}$)
    \item $E_a$: Energía de activación (kJ/mol)
    \item $R$: Constante de gases = 8.314 J/(mol·K)
    \item $T$: Temperatura (K)
\end{itemize}

\subsubsection{Requisitos}

Para un ajuste robusto, necesitas datos de \textbf{al menos 3 experimentos a diferentes temperaturas}:

\begin{itemize}
    \item Experimento a baja T (ej. 55°C)
    \item Experimento a T media (ej. 65°C)
    \item Experimento a alta T (ej. 75°C)
\end{itemize}

Todos procesados con el Módulo 1 (Sección 4.2).

\subsubsection{Método 1: Línea de Comandos}

\begin{lstlisting}[style=bashstyle]
python main.py --mode fit_params \
    --input variables_esterificacion_dataset.json \
    --output results/parameter_fitting/
\end{lstlisting}

\subsubsection{Método 2: Script de Ejemplo}

\begin{lstlisting}[style=bashstyle]
cp plantillas/ejemplo_02_ajustar_parametros.py .
# Editar archivo para especificar rutas de experimentos
python ejemplo_02_ajustar_parametros.py
\end{lstlisting}

\textbf{Editar en el script}:

\begin{lstlisting}[style=pythonstyle]
EXPERIMENTOS = [
    {'file': 'data/processed/exp_55C.csv', 'temperatura': 55.0},
    {'file': 'data/processed/exp_65C.csv', 'temperatura': 65.0},
    {'file': 'data/processed/exp_75C.csv', 'temperatura': 75.0},
]
\end{lstlisting}

\subsubsection{Resultados Generados}

\begin{enumerate}
    \item \textbf{Archivo JSON}: \texttt{results/parameter\_fitting/parametros\_ajustados.json}

    Contiene:
    \begin{lstlisting}[basicstyle=\ttfamily\small]
{
  "modelo": "1-step",
  "reversible": true,
  "parametros": {
    "A_forward": 2.98e10,  // min^-1
    "Ea_forward": 51.9,    // kJ/mol
    "A_reverse": 1.5e8,
    "Ea_reverse": 45.0
  },
  "metricas": {
    "R_squared": 0.9876,
    "RMSE": 1.234,
    "MAE": 0.987
  },
  "intervalos_confianza": { ... }
}
    \end{lstlisting}

    \item \textbf{Gráfica de ajuste}: \texttt{results/parameter\_fitting/ajuste\_parametros.png}

    Muestra conversión experimental vs modelo para cada temperatura.
\end{enumerate}

\subsubsection{Interpretación de Resultados}

\begin{infobox}{Evaluación de la Calidad del Ajuste}
\textbf{Coeficiente de determinación ($R^2$)}:
\begin{itemize}
    \item $R^2 > 0.95$: Excelente ajuste
    \item $0.90 < R^2 < 0.95$: Buen ajuste
    \item $0.80 < R^2 < 0.90$: Aceptable
    \item $R^2 < 0.80$: Ajuste pobre, revisar datos
\end{itemize}

\textbf{RMSE (Root Mean Square Error)}:
\begin{itemize}
    \item RMSE < 2\%: Muy bueno
    \item RMSE < 5\%: Aceptable
    \item RMSE > 10\%: Revisar modelo o datos
\end{itemize}
\end{infobox}

\textbf{Valores típicos de literatura para CaO}:
\begin{itemize}
    \item $E_a$: 50-80 kJ/mol
    \item $A$: $10^8$ - $10^{11}$ min$^{-1}$
\end{itemize}

Si tus valores ajustados difieren significativamente, considera:
\begin{enumerate}
    \item Calidad de los datos experimentales
    \item Tipo de aceite usado (puede afectar cinética)
    \item Pureza del catalizador CaO
    \item Modelo cinético (¿es apropiado el de 1 paso?)
\end{enumerate}

\subsection{Módulo 3: Optimización de Condiciones Operacionales}

\subsubsection{Objetivo}

Encontrar las condiciones óptimas de operación que maximizan la conversión de triglicéridos:

\begin{itemize}
    \item Temperatura (°C)
    \item Agitación (RPM)
    \item Concentración de catalizador (\% masa)
\end{itemize}

\subsubsection{Método 1: Línea de Comandos}

\begin{lstlisting}[style=bashstyle]
python main.py --mode optimize \
    --input results/parameter_fitting/parametros_ajustados.json \
    --output results/optimization/
\end{lstlisting}

\subsubsection{Método 2: Script de Ejemplo}

\begin{lstlisting}[style=bashstyle]
cp plantillas/ejemplo_03_optimizar.py .
python ejemplo_03_optimizar.py
\end{lstlisting}

\textbf{Configuración en el script}:

\begin{itemize}
    \item \texttt{PARAMETROS\_CINETICOS}: Usar valores del Módulo 2
    \item \texttt{C0}: Condiciones iniciales (mol/L)
    \item \texttt{TIEMPO\_REACCION}: Tiempo total (min)
    \item \texttt{BOUNDS}: Límites de optimización
\end{itemize}

\subsubsection{Algoritmos de Optimización Disponibles}

\begin{table}[H]
\centering
\caption{Algoritmos de optimización soportados}
\begin{tabular}{lll}
\toprule
\textbf{Algoritmo} & \textbf{Ventajas} & \textbf{Desventajas} \\
\midrule
Differential Evolution & Global, robusto & Lento (1-5 min) \\
SLSQP & Rápido & Local (depende de inicial) \\
Dual Annealing & Muy robusto & Muy lento (5-10 min) \\
Nelder-Mead & Simple & Local \\
\bottomrule
\end{tabular}
\end{table}

\textbf{Recomendación}: Usar \texttt{differential\_evolution} para búsqueda global.

\subsubsection{Resultados Generados}

\begin{enumerate}
    \item \textbf{Archivo JSON}: \texttt{results/optimization/condiciones\_optimas.json}

    \begin{lstlisting}[basicstyle=\ttfamily\small]
{
  "condiciones_optimas": {
    "temperature": 67.3,    // °C
    "rpm": 580.0,
    "catalyst_%": 3.2,
    "conversion_%": 94.7
  },
  "analisis_sensibilidad": {
    "temperature": +0.45,    // Sensibilidad positiva
    "rpm": +0.12,
    "catalyst_%": +0.38
  }
}
    \end{lstlisting}

    \item \textbf{Superficie de respuesta 3D}: \texttt{results/optimization/superficie\_respuesta.png}

    Gráfica 3D mostrando cómo varía la conversión con T y \% catalizador (a RPM fijo).

    \item \textbf{Tornado plot}: \texttt{results/optimization/analisis\_sensibilidad.png}

    Gráfica de barras mostrando qué variable tiene mayor impacto en la conversión.
\end{enumerate}

\subsubsection{Interpretación del Análisis de Sensibilidad}

\begin{infobox}{Sensibilidad}
Un valor de sensibilidad positivo significa que \textbf{aumentar} esa variable \textbf{aumenta} la conversión.

\textbf{Ejemplo}:
\begin{itemize}
    \item Sensibilidad(T) = +0.45 → Temperatura es la variable MÁS importante
    \item Sensibilidad(RPM) = +0.12 → RPM tiene efecto moderado
    \item Sensibilidad(Cat) = +0.38 → Catalizador muy importante
\end{itemize}

\textbf{Implicación práctica}: Si quieres mejorar conversión, prioriza ajustar temperatura, luego catalizador, luego RPM.
\end{infobox}

\subsection{Módulo 4: Comparación de Modelos}

\subsubsection{Objetivo}

Comparar estadísticamente los resultados entre diferentes modelos o configuraciones.

\subsubsection{Métricas Calculadas}

\begin{itemize}
    \item \textbf{RMSE}: Root Mean Square Error
    \item \textbf{MAE}: Mean Absolute Error
    \item \textbf{R$^2$}: Coeficiente de determinación
    \item \textbf{MAPE}: Mean Absolute Percentage Error
    \item \textbf{Pearson r}: Correlación de Pearson
\end{itemize}

\subsubsection{Ejecución}

\begin{lstlisting}[style=bashstyle]
python main.py --mode compare \
    --input results/ \
    --output results/comparison/
\end{lstlisting}

\subsubsection{Interpretación}

\begin{table}[H]
\centering
\caption{Criterios de evaluación de comparación}
\begin{tabular}{lll}
\toprule
\textbf{Métrica} & \textbf{Valor bueno} & \textbf{Interpretación} \\
\midrule
R$^2$ & > 0.95 & Excelente correlación \\
RMSE & < 2\% & Error muy bajo \\
MAPE & < 5\% & Error relativo aceptable \\
Pearson r & > 0.95 & Correlación lineal fuerte \\
\bottomrule
\end{tabular}
\end{table}

\begin{successbox}{Validación Exitosa}
Si R$^2 > 0.95$ y MAPE < 5\%, tu modelo muestra excelente precisión. Puedes confiar en los resultados del modelo para predicciones y optimización.
\end{successbox}

%==============================================================================
\section{Interpretación de Resultados}
%==============================================================================

\subsection{Gráficas Generadas}

\subsubsection{Conversión vs Tiempo}

\textbf{Ubicación}: \texttt{results/figures/conversion\_vs\_time.png}

\textbf{Cómo leer}:
\begin{itemize}
    \item Eje X: Tiempo de reacción (min)
    \item Eje Y: Conversión de TG (\%)
    \item Puntos: Datos experimentales
    \item Línea continua: Modelo ajustado
\end{itemize}

\textbf{Qué buscar}:
\begin{itemize}
    \item Pendiente inicial alta → Reacción rápida
    \item Plateau final → Equilibrio alcanzado
    \item Buen ajuste modelo-experimental → Línea pasa cerca de puntos
\end{itemize}

\subsubsection{Perfiles de Concentración}

\textbf{Ubicación}: \texttt{results/figures/concentration\_profiles.png}

Muestra evolución de concentraciones de todas las especies:
\begin{itemize}
    \item TG (disminuye)
    \item MeOH (disminuye)
    \item FAME (aumenta)
    \item GL (aumenta)
\end{itemize}

\textbf{Verificación de balance de masa}:
\begin{equation}
\Delta n_{TG} \times 3 = \Delta n_{FAME}
\end{equation}
\begin{equation}
\Delta n_{TG} = \Delta n_{GL}
\end{equation}

\subsubsection{Superficie de Respuesta 3D}

\textbf{Ubicación}: \texttt{results/optimization/superficie\_respuesta.png}

Visualización 3D de cómo la conversión varía con dos variables (ej. T vs \% catalizador).

\textbf{Cómo usar}:
\begin{itemize}
    \item Punto rojo = Condiciones óptimas
    \item Color caliente (rojo/amarillo) = Alta conversión
    \item Color frío (azul) = Baja conversión
    \item Identificar región óptima de operación
\end{itemize}


\textbf{Ubicación}: \texttt{results/comparison/parity\_plot.png}


\textbf{Interpretación}:
\begin{itemize}
    \item Puntos sobre línea y=x → Modelos coinciden
    \item Dispersión pequeña → Buena correlación
    \item Puntos alejados → Discrepancias (revisar)
\end{itemize}

\subsection{Archivos de Salida}

\subsubsection{Archivos JSON}

Formato estructurado, fácil de leer por programas:

\begin{lstlisting}[style=pythonstyle, caption=Leer JSON en Python]
import json
with open('results/parametros_ajustados.json', 'r') as f:
    data = json.load(f)
print(f"Ea = {data['parametros']['Ea_forward']} kJ/mol")
\end{lstlisting}

\subsubsection{Archivos Excel}

Múltiples hojas con resultados organizados:
\begin{itemize}
    \item Hoja 1: Parámetros ajustados
    \item Hoja 2: Métricas de ajuste
    \item Hoja 3: Datos experimentales vs modelo
    \item Hoja 4: Condiciones óptimas
\end{itemize}

Abre con Excel, LibreOffice Calc o Google Sheets.

\subsubsection{Archivos CSV}

Datos tabulares simples, compatibles con cualquier software de análisis de datos.

Ejemplo de uso en R:

\begin{lstlisting}[style=bashstyle]
> data <- read.csv("results/resultados_procesados.csv")
> plot(data$Tiempo_min, data$Conversion_pct)
\end{lstlisting}

%==============================================================================
\section{Casos de Uso Avanzados}
%==============================================================================

\subsection{Workflow Completo Automatizado}

Para ejecutar TODO el análisis de una vez:

\begin{lstlisting}[style=bashstyle]
cp plantillas/ejemplo_06_workflow_completo.py .
# Editar rutas de archivos
python ejemplo_06_workflow_completo.py
\end{lstlisting}

Este script ejecuta en secuencia:
\begin{enumerate}
    \item Procesamiento GC (todos los experimentos)
    \item Ajuste de parámetros
    \item Optimización de condiciones
    \item Generación de reportes
\end{enumerate}

\subsection{Uso Programático (Scripts Personalizados)}

Puedes importar los módulos en tus propios scripts Python:

\begin{lstlisting}[style=pythonstyle]
from src.models.kinetic_model import KineticModel

# Crear modelo
model = KineticModel(model_type='1-step', reversible=True, temperature=65.0)

# Condiciones iniciales
C0 = {'TG': 0.5, 'MeOH': 4.5, 'FAME': 0.0, 'GL': 0.0}

# Simular
results = model.simulate(t_span=(0, 120), C0=C0)

# Acceder a resultados
print(f"Conversión final: {results['conversion_%'][-1]:.2f}%")
\end{lstlisting}

Ver más ejemplos en README.md sección "Uso Programático".

\subsection{Integración con CFD (Ansys Fluent)}

\subsubsection{Especificaciones del Reactor 20L}

El documento \texttt{docs/reactor\_cfd\_specs.md} contiene especificaciones completas para simular el reactor en Ansys Fluent:

\textbf{Geometría actualizada}:
\begin{itemize}
    \item \textbf{Impulsor}: Hélice de flujo axial de 3 palas (NO Rushton)
    \item \textbf{Baffles}: SIN deflectores
    \item \textbf{Serpentín}: 10 vueltas helicoidales de acero inoxidable (d = 8 mm)
    \item Volumen: 20 L
    \item Diámetro tanque: 270 mm
    \item Altura líquido: 350 mm
\end{itemize}

\subsubsection{Pasos para CFD}

\begin{enumerate}
    \item Crear geometría CAD según especificaciones (sección Anexo A del documento CFD)
    \item Generar malla en Fluent Meshing (500k-1M celdas)
    \item Configurar modelos físicos:
    \begin{itemize}
        \item Turbulencia: k-ε RNG
        \item Especies: TG, MeOH, FAME, GL
        \item Reacción: UDF con parámetros ajustados
    \end{itemize}
    \item Ejecutar simulación
    \item Post-procesar: campos de velocidad, temperatura, concentración
\end{enumerate}

\textbf{UDF disponible}: Ver sección 8 de \texttt{reactor\_cfd\_specs.md} para código C completo.

\textbf{Script PyFluent}: Sección 10 tiene script Python para automatizar setup.

\subsection{Batch Processing (Múltiples Experimentos)}

Para procesar múltiples experimentos automáticamente:

\begin{lstlisting}[style=pythonstyle]
import glob
from src.data_processing.gc_processor import GCProcessor

processor = GCProcessor()

# Procesar todos los CSV en data/raw/
for file in glob.glob('data/raw/*.csv'):
    print(f"Procesando: {file}")
    data = processor.load_from_csv(file)
    results = processor.process_time_series(data, C_TG0=0.5)

    # Guardar resultados
    output_name = file.replace('raw', 'processed')
    results.to_csv(output_name, index=False)

print("Batch processing completado")
\end{lstlisting}

%==============================================================================
\section{Troubleshooting (Resolución de Problemas)}
%==============================================================================

\subsection{Errores Comunes de Instalación}

\subsubsection{Error: "ModuleNotFoundError: No module named 'numpy'"}

\textbf{Causa}: Dependencias no instaladas.

\textbf{Solución}:
\begin{lstlisting}[style=bashstyle]
pip install -r requirements.txt
\end{lstlisting}

\subsubsection{Error: "pip: command not found"}

\textbf{Causa}: pip no está instalado o no está en PATH.

\textbf{Solución}:
\begin{itemize}
    \item Linux/Mac: \texttt{sudo apt install python3-pip} o \texttt{brew install python}
    \item Windows: Reinstalar Python marcando "Add to PATH"
\end{itemize}

\subsubsection{Error al instalar pywin32 en Linux/Mac}

\textbf{Causa}: pywin32 solo funciona en Windows.


\subsection{Errores de Datos}

\subsubsection{Error: "KeyError: 'TG'"}

\textbf{Causa}: Nombre de compuesto incorrecto en CSV.

\textbf{Solución}: Verificar que los nombres de compuestos sean exactamente:
\texttt{TG}, \texttt{MeOH}, \texttt{FAME}, \texttt{GL}, \texttt{Estándar\_Interno}

(Respetar mayúsculas/minúsculas y guiones bajos)

\subsubsection{Error: "FileNotFoundError"}

\textbf{Causa}: Ruta de archivo incorrecta o archivo no existe.

\textbf{Solución}:
\begin{enumerate}
    \item Verificar que el archivo existe: \texttt{ls data/raw/mi\_archivo.csv}
    \item Usar rutas absolutas si es necesario
    \item Verificar permisos de lectura del archivo
\end{enumerate}

\subsubsection{Error: "JSON decode error"}

\textbf{Causa}: Archivo JSON mal formado.

\textbf{Solución}:
\begin{enumerate}
    \item Copiar contenido del JSON
    \item Pegar en validador online: \url{https://jsonlint.com}
    \item Corregir errores indicados (comas faltantes, comillas, etc.)
\end{enumerate}

\subsection{Errores de Convergencia}

\subsubsection{Advertencia: "Integration failed to converge"}

\textbf{Causa}: Parámetros cinéticos extremos o condiciones iniciales inválidas.

\textbf{Solución}:
\begin{enumerate}
    \item Verificar valores de $A$ y $E_a$ (deben estar en rangos razonables)
    \item Verificar $C_0$ (concentraciones positivas, relación molar coherente)
    \item Reducir tolerancias: modificar \texttt{rtol} y \texttt{atol} en código
\end{enumerate}

\subsubsection{Error: "Optimization did not converge"}

\textbf{Causa}: Algoritmo de optimización no encontró óptimo en iteraciones permitidas.

\textbf{Solución}:
\begin{enumerate}
    \item Aumentar \texttt{maxiter} (ej. de 100 a 500)
    \item Probar algoritmo diferente (ej. \texttt{dual\_annealing})
    \item Ajustar límites de optimización (pueden ser muy estrechos)
\end{enumerate}




\textbf{Solución}:
\begin{enumerate}
\end{enumerate}


\textbf{Causa}: Permisos insuficientes.

\textbf{Solución}: Ejecutar Python/script como administrador.


\textbf{Causa}: Condiciones termodinámicas inválidas o parámetros extremos.

\textbf{Solución}:
\begin{enumerate}
    \item Revisar paquete termodinámico (UNIFAC es apropiado para biodiésel)
    \item Simplificar modelo (usar reacción más simple)
    \item Revisar conversión batch→continuo en \texttt{data\_sync.py}
\end{enumerate}

\subsection{Problemas de Rendimiento}

\subsubsection{Optimización muy lenta}

\textbf{Soluciones}:
\begin{enumerate}
    \item Reducir \texttt{maxiter} (sacrifica precisión por velocidad)
    \item Usar algoritmo más rápido: \texttt{SLSQP} o \texttt{Nelder-Mead}
    \item Reducir resolución de superficie de respuesta
    \item Usar menos puntos en simulación (reducir \texttt{n\_points})
\end{enumerate}

\subsubsection{Memoria insuficiente}

\textbf{Soluciones}:
\begin{enumerate}
    \item Procesar experimentos de uno en uno (no batch)
    \item Reducir resolución de gráficas
    \item Cerrar otras aplicaciones
    \item Si es en CFD: usar malla más gruesa
\end{enumerate}

\subsection{Obtener Ayuda Adicional}

\begin{infobox}{Recursos de Soporte}
\begin{itemize}
    \item \textbf{Documentación principal}: \texttt{README.md}
    \item \textbf{Documentación académica}: \texttt{docs/documento\_latex.tex}
    \item \textbf{Especificaciones CFD}: \texttt{docs/reactor\_cfd\_specs.md}
    \item \textbf{Ejemplos de código}: Carpeta \texttt{plantillas/}
    \item \textbf{Issues en GitHub}: Reportar bugs o pedir features
\end{itemize}
\end{infobox}

%==============================================================================
\section{Apéndices}
%==============================================================================

\subsection{Apéndice A: Comandos Rápidos de Referencia}

\begin{table}[H]
\centering
\caption{Comandos principales del sistema}
\begin{tabular}{p{0.45\textwidth}p{0.45\textwidth}}
\toprule
\textbf{Acción} & \textbf{Comando} \\
\midrule
Activar entorno virtual (Linux/Mac) & \texttt{source venv/bin/activate} \\
Activar entorno virtual (Windows) & \texttt{venv\textbackslash Scripts\textbackslash activate} \\
Procesar datos GC & \texttt{python main.py --mode process\_gc --input data/raw/exp.csv} \\
Ajustar parámetros & \texttt{python main.py --mode fit\_params --input exp.json} \\
Optimizar condiciones & \texttt{python main.py --mode optimize --output results/} \\
Comparar modelos & \texttt{python main.py --mode compare} \\
Workflow completo & \texttt{python ejemplo\_06\_workflow\_completo.py} \\
\bottomrule
\end{tabular}
\end{table}

\subsection{Apéndice B: Estructura de Directorios}

\begin{lstlisting}[basicstyle=\ttfamily\small]
mod_esterificacion/
├── data/
│   ├── raw/                  # Datos GC-FID crudos (CSV)
│   ├── processed/            # Datos procesados
│   └── literature/           # Datos de literatura
├── src/
│   ├── data_processing/      # Módulos de procesamiento
│   ├── models/               # Modelos cinéticos
│   ├── optimization/         # Optimización
│   ├── visualization/        # Gráficas
│   └── utils/                # Utilidades
├── results/                  # Resultados generados
│   ├── figures/              # Gráficas PNG/PDF
│   ├── reports/              # Reportes Excel/PDF
│   └── exports/              # Exportaciones JSON
├── docs/                     # Documentación
│   ├── documento_latex.tex   # Doc académico
│   ├── reactor_cfd_specs.md  # Especificaciones CFD
│   └── tutorial_uso_sistema.tex  # Este tutorial
├── plantillas/               # Plantillas listas para usar
│   ├── plantilla_datos_gc.csv
│   ├── plantilla_experimento.json
│   ├── plantilla_config.yaml
│   └── ejemplo_XX_*.py       # 6 scripts de ejemplo
├── main.py                   # CLI principal
├── requirements.txt          # Dependencias Python
├── README.md                 # Documentación principal
└── variables_esterificacion_dataset.json  # Esquema de datos
\end{lstlisting}

\subsection{Apéndice C: Glosario de Términos}

\begin{description}
    \item[FAME] Fatty Acid Methyl Esters (Ésteres Metílicos de Ácidos Grasos) - Componente principal del biodiésel
    \item[TG] Triglicéridos - Aceite vegetal (reactivo)
    \item[DG] Diglicéridos - Intermediario de reacción
    \item[MG] Monoglicéridos - Intermediario de reacción
    \item[GL] Glicerol - Subproducto de reacción
    \item[GC-FID] Gas Chromatography - Flame Ionization Detector (Cromatografía de Gases con Detector de Ionización de Llama)
    \item[CFD] Computational Fluid Dynamics (Dinámica de Fluidos Computacional)
    \item[Ea] Energía de activación (kJ/mol)
    \item[A] Factor pre-exponencial de Arrhenius (min$^{-1}$)
    \item[RMSE] Root Mean Square Error (Error Cuadrático Medio)
    \item[MAE] Mean Absolute Error (Error Absoluto Medio)
    \item[R$^2$] Coeficiente de determinación
    \item[MRF] Multiple Reference Frame (Marco de Referencia Múltiple)
    \item[UDF] User-Defined Function (Función Definida por Usuario) en Fluent
    \item[VOF] Volume of Fluid (Modelo multifásico)
\end{description}

\subsection{Apéndice D: Referencias Bibliográficas}

\begin{enumerate}
    \item Kouzu, M., et al. (2008). "Heterogeneous catalysis of calcium oxide used for transesterification of soybean oil with refluxing methanol". \textit{Applied Catalysis A}, 355(1-2), 94-99.

    \item Sharma, Y.C., et al. (2011). "Optimization of parameters for biodiesel production from Jatropha curcas oil". \textit{Fuel}, 90(3), 1083-1088.

    \item Paul, E.L., Atiemo-Obeng, V.A., \& Kresta, S.M. (2004). \textit{Handbook of Industrial Mixing}. Wiley.

    \item Nienow, A.W. (1997). "On impeller circulation and mixing effectiveness in the turbulent flow regime". \textit{Chemical Engineering Science}, 52(15), 2557-2565.

    \item Ansys Inc. (2023). \textit{Ansys Fluent Theory Guide}. Release 2023 R1.
\end{enumerate}

%==============================================================================
% FIN DEL DOCUMENTO
%==============================================================================

\vspace{2cm}

\begin{center}
\rule{\textwidth}{0.4pt}

\vspace{0.5cm}

\textbf{\Large ¡Gracias por usar el Sistema de Modelado de Esterificación!}

\vspace{0.3cm}

Si este tutorial te fue útil, considera:
\begin{itemize}
    \item Contribuir al proyecto en GitHub
    \item Reportar bugs o sugerir mejoras
    \item Compartir tus resultados con la comunidad
\end{itemize}

\vspace{0.5cm}

\textit{Versión 1.0 -- \today}

\rule{\textwidth}{0.4pt}
\end{center}

\end{document}
