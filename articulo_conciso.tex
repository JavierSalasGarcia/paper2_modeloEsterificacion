\documentclass{RICI}

% Paquetes adicionales necesarios
\usepackage{longtable}
\usepackage{graphicx}
\usepackage{multicol}
\usepackage{mathtools}
\usepackage{amsmath}
\usepackage{booktabs}
\usepackage{multirow}
\usepackage{microtype}

%%%%%%%%%%%%%%%%%%%%%%%%%%%%%%%%%%%%
% Configuraciones editoriales
\fecharec{22 de noviembre de 2025}
\fechaace{-- de ---- de 2026}
\VolR{4}
\NumR{1}
\meses{enero-junio 2026}
\pagfinal{1}
%%%%%%%%%%%%%%%%%%%%%%%%%%%%%%%%%%%%

\autencabez{Salas-García et al.}

\begin{document}

% Títulos en ambos idiomas
\title{Sistema Unificado Open-Source para Modelado Cinético de Transesterificación: Versatilidad y Adaptabilidad mediante Casos de Uso Especializados}

\titleEng{Unified Open-Source System for Kinetic Modeling of Transesterification: Versatility and Adaptability through Specialized Use Cases}

% Información de los autores
\author{Javier Salas-García\authornote{1}\corrAuthor\orcidlink{0000-0000-0000-0000}, Miguel Moran Gonzalez\authornote{1}\orcidlink{0000-0000-0000-0000}, María Dolores Durán García\authornote{1}\orcidlink{0000-0000-0000-0000}, Rosa Romero Romero\authornote{2}\orcidlink{0000-0000-0000-0000}, Reyna Natividad Rangel\authornote{2}\orcidlink{0000-0000-0000-0000}}

\authoraddress{1}{Facultad de Ingeniería, Universidad Autónoma del Estado de México (UAEMEX), Toluca, México.}
\authoraddress{2}{Centro Conjunto de Investigación en Química Sustentable UAEM–UNAM (CCIQS UAEM-UNAM), Toluca, México.}

\emailCorr{proyectos@javiersalasg.com}

\maketitle

% Resumen en español
\begin{resumen}
\noindent El modelado cinético de transesterificación para producción de biodiésel requiere herramientas que combinen accesibilidad y capacidad analítica. Este trabajo presenta un sistema unificado open-source que coordina módulos especializados mediante interfaz de línea de comandos, demostrando versatilidad a través de seis casos de uso. El sistema integra procesamiento de datos cromatográficos, ajuste de parámetros cinéticos, optimización multi-objetivo, comparación de modelos mecanísticos, análisis de sensibilidad y escalado de reactores. Los parámetros cinéticos calibrados con datos de Kouzu et al. alcanzan $R^2=0.9844$ y RMSE=3.85\%. Los casos demuestran: (1) procesamiento GC-FID en 3.2~s reduciendo 20 pasos manuales, (2) calibración convergiendo en 24.7~s con intervalos de confianza automáticos, (3) optimización alcanzando 93\% conversión en 112~s, (4) comparación de modelos 1-paso vs 3-pasos con diferencia <0.3\%, (5) análisis factorial (192 simulaciones, 87~s) identificando temperatura como variable crítica (42.1\% contribución), (6) diseño de reactor piloto 20~L desde 350~mL con diferencia de conversión <0.1\%. El sistema bajo licencia MIT proporciona alternativa accesible al software comercial (\$50,000--\$100,000 anuales), democratizando herramientas de simulación para instituciones con recursos limitados.

\palabrasclave{\emph{Biodiésel, Transesterificación, Modelado cinético, Software open-source, Sistema unificado, Python}}
\end{resumen}

% Abstract en inglés
\begin{abstract}
\noindent Kinetic modeling of transesterification for biodiesel production requires tools combining accessibility and analytical capability. This work presents a unified open-source system coordinating specialized modules through command-line interface, demonstrating versatility via six use cases. The system integrates chromatographic data processing, kinetic parameter fitting, multi-objective optimization, mechanistic model comparison, sensitivity analysis, and reactor scale-up. Kinetic parameters calibrated with Kouzu et al. data achieve $R^2=0.9844$ and RMSE=3.85\%. Cases demonstrate: (1) GC-FID processing in 3.2~s reducing 20 manual steps, (2) calibration converging in 24.7~s with automatic confidence intervals, (3) optimization reaching 93\% conversion in 112~s, (4) 1-step vs 3-step model comparison with difference <0.3\%, (5) factorial analysis (192 simulations, 87~s) identifying temperature as critical variable (42.1\% contribution), (6) 20~L pilot reactor design from 350~mL with conversion difference <0.1\%. The MIT-licensed system provides accessible alternative to commercial software (\$50,000--\$100,000 annually), democratizing simulation tools for resource-limited institutions.

\keywords{\emph{Biodiesel, Transesterification, Kinetic modeling, Open-source software, Unified system, Python}}
\end{abstract}

%========================================================================
% INTRODUCCIÓN
%========================================================================
\section{Introducción}

La transesterificación de aceites vegetales catalizada por CaO constituye una ruta prometedora para producción de biodiésel sustentable~\cite{Kouzu2008,Atadashi2013}. El modelado cinético permite predecir conversiones, optimizar parámetros y diseñar reactores, pero las herramientas disponibles presentan limitaciones significativas. El software comercial como Aspen Plus~\cite{AspenPlus2024} y COMSOL~\cite{COMSOL2024} requiere licencias de \$50,000--\$100,000 anuales, inaccesibles para instituciones de países en desarrollo. Herramientas open-source como BioSTEAM~\cite{CortesPena2020}, SKiMpy~\cite{Saa2023}, Cantera~\cite{Goodwin2023Cantera} y PyOMO~\cite{Hart2017PyOMO} son gratuitas pero requieren experiencia avanzada en programación y configuración extensiva.

Un trabajo previo~\cite{SalasGarcia2025Informaticae} presentó 13 prácticas educativas con scripts independientes para aprendizaje progresivo. El presente sistema adopta filosofía diferente: programa unificado operado mediante parámetros de modo que facilita productividad mediante integración. Los usuarios ejecutan análisis específicos sin necesidad de comprender implementación completa del sistema.

Este trabajo demuestra versatilidad mediante seis casos de uso representativos cubriendo operaciones comunes en investigación de biodiésel. Los parámetros cinéticos fueron calibrados con datos de Kouzu et al.~\cite{Kouzu2008} y validados con estudios independientes. El sistema está disponible bajo licencia MIT, eliminando barreras económicas y promoviendo reproducibilidad científica.

%========================================================================
% METODOLOGÍA
%========================================================================
\section{Metodología}

\subsection{Arquitectura del Sistema}

El sistema adopta arquitectura modular coordinada por \texttt{main.py} con cinco capas: (1) procesamiento de datos (GC-FID, JSON, propiedades), (2) modelado (cinéticas 1-paso y 3-pasos, Arrhenius), (3) optimización (regresión no lineal, evolución diferencial, ANOVA), (4) visualización, (5) aplicación (CLI). El parámetro \texttt{--mode} determina la operación: \texttt{process\_gc}, \texttt{fit\_params}, \texttt{optimize}, \texttt{compare}, \texttt{sensitivity}, \texttt{scaleup}.

\subsection{Modelos Cinéticos}

\subsubsection{Modelo 1-Paso}
Representa la transesterificación como reacción global reversible~\cite{Kouzu2008}, suficiente para predicción de conversión final~\cite{Aziz2025,Likozar2021}:
\begin{equation}
\text{TG} + 3\,\text{MeOH} \xrightleftharpoons[k_{-1}]{k_{1}} 3\,\text{FAME} + \text{GL}
\label{eq:modelo_1paso}
\end{equation}
\begin{equation}
r = k_1(T)\,[\text{TG}][\text{MeOH}]^3 - k_{-1}(T)\,[\text{FAME}]^3[\text{GL}]
\end{equation}
con $k_i(T) = A_i \exp(-E_{a,i}/RT)$, donde $A_i$ es el factor preexponencial [L/(mol·min)], $E_{a,i}$ la energía de activación [J/mol], $R=8.314$~J/(mol·K) y $T$ [K].

\subsubsection{Modelo 3-Pasos}
Descompone la reacción en tres etapas consecutivas capturando intermediarios DG y MG~\cite{Likozar2021,Hajjari2022}:
\begin{align}
\text{TG} + \text{MeOH} &\xrightleftharpoons[k_{-1}]{k_{1}} \text{DG} + \text{FAME} \\
\text{DG} + \text{MeOH} &\xrightleftharpoons[k_{-2}]{k_{2}} \text{MG} + \text{FAME} \\
\text{MG} + \text{MeOH} &\xrightleftharpoons[k_{-3}]{k_{3}} \text{GL} + \text{FAME}
\end{align}
Requiere seis parámetros cinéticos y proporciona información detallada sobre selectividad y pureza.

\subsection{Calibración y Validación}

Los parámetros del modelo 1-paso fueron ajustados con datos de Kouzu et al.~\cite{Kouzu2008} (60, 65, 70, 75~°C, 28 puntos) mediante Levenberg-Marquardt minimizando:
\begin{equation}
\text{SSR} = \sum_{i=1}^{28} \left( X_{\text{exp},i} - X_{\text{mod},i}(A, E_a) \right)^2
\end{equation}
Resultados: $A=8.0\times10^5$~L/(mol·min), $E_a=50.0$~kJ/mol, $R^2=0.9844$, RMSE=3.85\%. El $R^2$ es comparable con Balajii y Niju~\cite{Balajii2021} ($R^2=0.9886$), y $E_a$ está dentro del rango 35--68~kJ/mol reportado~\cite{Hajjari2022}. Validación cruzada con Liu et al.~\cite{Liu2008} y Granados et al.~\cite{Granados2007} confirma errores <1\%.

\subsection{Casos de Uso}

La Tabla~\ref{tab:casos_modulos} muestra la relación entre casos y módulos empleados.

\begin{table}[htbp]
\centering
\caption{Relación entre casos de uso y módulos del sistema.}
\label{tab:casos_modulos}
\small
\begin{tabular}{lccccccc}
\toprule
\textbf{Módulo} & \textbf{C1} & \textbf{C2} & \textbf{C3} & \textbf{C4} & \textbf{C5} & \textbf{C6} \\
\midrule
\texttt{gc\_processor} & $\times$ & -- & -- & -- & -- & -- \\
\texttt{parameter\_fitting} & -- & $\times$ & -- & -- & -- & -- \\
\texttt{optimizer} & -- & -- & $\times$ & -- & -- & -- \\
\texttt{comparison} & -- & -- & -- & $\times$ & -- & -- \\
\texttt{kinetic\_model} & $\times$ & $\times$ & $\times$ & $\times$ & $\times$ & $\times$ \\
Diseño factorial/ANOVA & -- & -- & -- & -- & $\times$ & -- \\
Escalado/Hidrodinámica & -- & -- & -- & -- & -- & $\times$ \\
\bottomrule
\end{tabular}
\end{table}

\textbf{Caso 1 (GC-FID):} Procesa datos cromatográficos convirtiendo áreas de picos en concentraciones mediante factores de respuesta y estándar interno (decano, 0.1~mol/L). Genera estadísticas descriptivas y visualizaciones.

\textbf{Caso 2 (Ajuste):} Calibra parámetros mediante Levenberg-Marquardt con datos a múltiples temperaturas. Integración numérica vía Radau (SciPy). Reporta parámetros con intervalos de confianza al 95\% y genera gráficos de validación y Arrhenius.

\textbf{Caso 3 (Optimización):} Identifica condiciones óptimas (T, relación molar, catalizador, agitación) mediante evolución diferencial maximizando conversión. Rangos: T=50--80~°C, relación=3:1--15:1, CaO=0.5--5\%, agitación=200--800~rpm.

\textbf{Caso 4 (Comparación):} Evalúa modelo 1-paso vs 3-pasos bajo condiciones idénticas (60~°C, 6:1, 0.5~mol/L TG, 120~min). Compara conversión, tiempo de cómputo y detalle mecanístico.

\textbf{Caso 5 (Sensibilidad):} Ejecuta diseño factorial completo $4\times4\times4\times3=192$ simulaciones (T: 55,60,65,70~°C; relación: 4,6,8,10; catalizador: 0.5,1.0,1.5,2.0\%; agitación: 300,500,700~rpm). ANOVA identifica variables críticas.

\textbf{Caso 6 (Escalado):} Diseña reactor piloto 20~L desde 350~mL mediante criterios de similitud hidrodinámica: $Np$ constante, $P/V$ constante, velocidad de punta constante, tiempo de mezclado constante. Valida mediante simulaciones cinéticas.

\subsection{Implementación}

Sistema implementado en Python $\geq$3.8 con NumPy~\cite{Harris2020NumPy}, SciPy~\cite{Virtanen2020SciPy} (Radau, Levenberg-Marquardt, evolución diferencial), Pandas, Matplotlib y lmfit (intervalos de confianza). Configuraciones en JSON. Scripts bash para automatización.

%========================================================================
% RESULTADOS
%========================================================================
\section{Resultados}

\subsection{Caso 1: Procesamiento GC-FID}

El sistema procesó 28 filas (4 compuestos, 7 tiempos) en 3.2~s. La Figura~\ref{fig:caso1} muestra conversión alcanzando 92.1\% a 120~min (desviación 0.1\% vs experimental), perfiles de concentración ([TG]: 0.5$\to$0.04~mol/L, [FAME]: 0$\to$1.4~mol/L, [GL]: 0$\to$0.46~mol/L), rendimiento FAME 92\%, y balance de masa. Estadísticas: conversión 92.1\%, RMSE 2.8\%, IC95\%=$\pm$5.5\%, cero outliers. La conversión es consistente con Ahmed et al.~\cite{Ahmed2021} (94\%, CaO nano, 60~°C, 120~min), Niju et al.~\cite{Niju2024} (95\%, CaO/hectorita) y Adepoju et al.~\cite{Adepoju2020} (94.5\%, 60~°C), confirmando resultados realistas. El método manual requiere 20 pasos y 15--20~min; el automatizado ejecuta en 3.2~s con reproducibilidad perfecta.

\begin{figure}[htbp]
\centering
\includegraphics[width=0.95\textwidth]{figuras/resultados_gc_visualizacion.png}
\caption{Procesamiento GC-FID: conversión, perfiles de concentración, rendimiento FAME, balance de masa.}
\label{fig:caso1}
\end{figure}

\subsection{Caso 2: Calibración}

Convergencia en 24.7~s (147 evaluaciones): $A=8.02\times10^5$~L/(mol·min) (IC95\%: 7.61--8.43$\times10^5$), $E_a=49.8$~kJ/mol (IC95\%: 48.6--51.0), $R^2=0.9844$, RMSE=3.85\%, MAE=3.12\%. La Figura~\ref{fig:caso2_ajuste} muestra ajuste a 60--75~°C: conversiones 92\%, 95\%, 97\%, 98\% a 120~min respectivamente, con residuos aleatorios pequeños. La Figura~\ref{fig:caso2_arrhenius} confirma linealidad Arrhenius ($R^2=0.998$). Aspen Plus requiere 15--20 pasos de configuración vs comando único aquí; intervalos de confianza automáticos vs post-procesamiento externo.

\begin{figure}[htbp]
\centering
\includegraphics[width=0.85\textwidth]{figuras/ajuste_experimental_vs_modelo.png}
\caption{Ajuste modelo 1-paso a datos Kouzu et al. (2008) a cuatro temperaturas.}
\label{fig:caso2_ajuste}
\end{figure}

\begin{figure}[htbp]
\centering
\includegraphics[width=0.65\textwidth]{figuras/arrhenius_plot.png}
\caption{Gráfico Arrhenius confirmando linealidad ($R^2=0.998$).}
\label{fig:caso2_arrhenius}
\end{figure}

\subsection{Caso 3: Optimización}

Convergencia en 112~s: T=65~°C, relación=6:1, CaO=0.5\%, agitación=200~rpm, conversión=93.04\%. Consistente con Piker et al.~\cite{Piker2024} (94\%, fotocatálisis, 60~°C) y Banani et al.~\cite{Banani2025} (>95\%, ML optimization). La Figura~\ref{fig:caso3_convergencia} muestra evolución: inicio 75\%, convergencia 93\% tras iteración 100. PyOMO requeriría 1--2~h de configuración vs 112~s aquí.

\begin{figure}[htbp]
\centering
\includegraphics[width=0.7\textwidth]{figuras/convergencia_optimizacion.png}
\caption{Convergencia evolución diferencial: 93.04\% en iteración 112.}
\label{fig:caso3_convergencia}
\end{figure}

\subsection{Caso 4: Comparación Modelos}

Condiciones: 60~°C, 6:1, 0.5~mol/L TG, 120~min. Modelo 1-paso: 92.1\%, 0.47~s. Modelo 3-pasos: 91.8\%, 1.38~s. Diferencia: 0.3\% (factor tiempo: 2.9$\times$). La Figura~\ref{fig:caso4_conversion} muestra curvas superpuestas. La Figura~\ref{fig:caso4_perfiles} compara perfiles: especies comunes similares; modelo 3-pasos predice DG y MG. La Figura~\ref{fig:caso4_intermediarios} muestra DG máximo 0.15~mol/L (30~min), MG máximo 0.12~mol/L (40~min), información crítica para pureza ausente en modelo 1-paso. Resultados confirman análisis de Likozar et al.~\cite{Likozar2021} y criterios de Aziz et al.~\cite{Aziz2025} (diferencia <5\% justifica modelo simple).

\begin{figure}[htbp]
\centering
\includegraphics[width=0.7\textwidth]{figuras/conversion_1paso_vs_3pasos.png}
\caption{Conversión 1-paso vs 3-pasos: curvas prácticamente superpuestas.}
\label{fig:caso4_conversion}
\end{figure}

\begin{figure}[htbp]
\centering
\includegraphics[width=0.85\textwidth]{figuras/perfiles_1paso_vs_3pasos.png}
\caption{Perfiles de concentración: modelo 3-pasos incluye intermediarios DG, MG.}
\label{fig:caso4_perfiles}
\end{figure}

\begin{figure}[htbp]
\centering
\includegraphics[width=0.65\textwidth]{figuras/intermediarios_DG_MG.png}
\caption{Intermediarios DG y MG: acumulación secuencial informa sobre selectividad.}
\label{fig:caso4_intermediarios}
\end{figure}

\subsection{Caso 5: Sensibilidad}

192 simulaciones en 87~s (0.45~s/sim). ANOVA: temperatura 42.1\%, relación molar 28.3\%, catalizador 21.5\%, agitación 8.1\%. Consistente con Santana et al.~\cite{Santana2024} (temperatura parámetro dominante) y Niju et al.~\cite{Niju2024} (dependencia exponencial 50--70~°C). Temperatura domina con primeras dos variables acumulando 70.4\%, primeras tres 91.9\%. Efectos principales: T monotónico (78\%@55~°C $\to$ 96\%@70~°C), relación molar positivo con rendimientos decrecientes, catalizador saturación >1.5\%, agitación débil (300--700~rpm suficiente). Interacción T$\times$relación significativa ($p<0.001$): efecto relación se amplifica a T alta. Superficie 3D muestra conversión >95\% en esquina T alta/relación alta, gradiente más pronunciado en dirección T. Implicaciones: priorizar control T (PID), relación 6--8, catalizador 1.0--1.5\%, agitación régimen turbulento.

\begin{figure}[htbp]
\centering
\includegraphics[width=0.8\textwidth]{figuras/efectos_principales.png}
\caption{Efectos principales: T y relación molar fuertes y monotónicos; catalizador saturación; agitación débil.}
\label{fig:caso5_efectos}
\end{figure}

\begin{figure}[htbp]
\centering
\includegraphics[width=0.6\textwidth]{figuras/interacciones_T_vs_RM.png}
\caption{Interacción T$\times$relación: efecto relación amplificado a T altas.}
\label{fig:caso5_interacciones}
\end{figure}

\begin{figure}[htbp]
\centering
\includegraphics[width=0.7\textwidth]{figuras/superficie_respuesta_3D.png}
\caption{Superficie 3D: conversión vs T y relación; gradiente pronunciado en dirección T.}
\label{fig:caso5_superficie}
\end{figure}

\subsection{Caso 6: Escalado}

Factor volumétrico: 57$\times$ (0.35~L$\to$20~L). Cuatro criterios: $Np$ constante (113~rpm, Re=24,500), $P/V$ constante (153~rpm, Re=33,200), velocidad punta (103~rpm, Re=22,300), tiempo mezclado (400~rpm, Re=86,800 prohibitivo). Selección: $P/V$ constante. Especificaciones piloto: 20~L, diámetro 310.6~mm, altura 271.4~mm (H/D=0.875), impulsor cinta helicoidal diámetro 116.4~mm (D/T=0.375), 153~rpm, Re=33,200 (turbulento), potencia 12.8~W (0.64~W/L). Validación cinética (60~°C, 6:1, 1\% CaO, 60~min): laboratorio 72.1\%, piloto 72.0\%, diferencia 0.1\% (<5\% criterio). Fregolente et al.~\cite{Fregolente2023} confirman preservación de desempeño cinético mediante similitud hidrodinámica.

\begin{figure}[htbp]
\centering
\includegraphics[width=0.75\textwidth]{figuras/comparacion_criterios_escalado.png}
\caption{Comparación criterios escalado: $P/V$ constante seleccionado (balance mezclado/eficiencia).}
\label{fig:caso6_comparacion}
\end{figure}

\begin{figure}[htbp]
\centering
\includegraphics[width=0.65\textwidth]{figuras/diagrama_reactor_piloto_3D.png}
\caption{Diagrama 3D reactor piloto 20~L: cinta helicoidal preserva geometría.}
\label{fig:caso6_diagrama}
\end{figure}

\begin{figure}[htbp]
\centering
\includegraphics[width=0.7\textwidth]{figuras/validacion_escalado.png}
\caption{Validación escalado: curvas laboratorio/piloto prácticamente indistinguibles.}
\label{fig:caso6_validacion}
\end{figure}

%========================================================================
% DISCUSIÓN
%========================================================================
\section{Discusión}

\subsection{Ventajas del Sistema}

La arquitectura unificada proporciona ventajas cuantificables: (1) Reducción complejidad: GC 20 pasos$\to$1 comando (3.2~s), ajuste 15--20 pasos Aspen$\to$1 comando (24.7~s), optimización 1--2~h PyOMO$\to$112~s. (2) Eliminación barreras económicas: MIT license vs \$50,000--\$100,000/año Aspen Plus. (3) Validación incorporada: parámetros pre-calibrados ($R^2=0.9844$) vs usuarios deben encontrar/validar manualmente. (4) Integración flujos: GC$\to$ajuste$\to$optimización$\to$sensibilidad$\to$escalado sin cambiar herramientas/formatos.

\subsection{Comparación con Alternativas}

Software comercial: capacidades extensivas, barreras costo/aprendizaje. Open-source genérico: BioSTEAM~\cite{CortesPena2020} (31,000 diseños <50~min, requiere Python/termodinámica), Cantera~\cite{Goodwin2023Cantera} (XML complejo), SKiMpy~\cite{Saa2023} (orientado biología). Sistema especializado combina ventajas: gratuito, simple, validación incorporada, documentación biodiésel. Curva aprendizaje: competencia básica 1--2~h, avanzada 1--2~días vs semanas (Aspen) o meses (PyOMO). Transparencia código facilita validación científica y extensibilidad vs algoritmos propietarios.

\subsection{Limitaciones}

(1) Especialización CaO: adaptación requerida para otros catalizadores/reacciones (arquitectura modular facilita). (2) CLI: barrera para usuarios sin experiencia terminal (GUI web en desarrollo futuro). (3) Validación basada en catalizadores heterogéneos similares: extensión a homogéneos/condiciones extremas requiere validación adicional. (4) Escalado: correlaciones semi-empíricas, no CFD completo; validación CFD/experimental recomendada para proyectos críticos.

%========================================================================
% CONCLUSIONES
%========================================================================
\section{Conclusiones}

Sistema unificado open-source para modelado cinético de transesterificación demuestra versatilidad mediante seis casos representativos. Resultados cuantifican ventajas vs software comercial/open-source genérico: reducción complejidad, eliminación barreras económicas, integración flujos, validación incorporada. Casos: (C1) GC 3.2~s, (C2) ajuste 24.7~s $R^2=0.9844$, (C3) optimización 93\% 112~s, (C4) comparación 1-paso/3-pasos diferencia 0.3\%, (C5) factorial 192 sims 87~s T=42.1\% contribución, (C6) escalado 350~mL$\to$20~L diferencia 0.1\%.

Sistema disponible bajo MIT proporcionando alternativa accesible (\$50,000--\$100,000/año software comercial). Arquitectura modular permite extensibilidad; documentación especializada reduce curvas aprendizaje. Democratiza herramientas para instituciones con recursos limitados, facilitando investigación energías renovables y desarrollo sustentable.

Impacto esperado: educación (aprendizaje sin inversiones prohibitivas), investigación (prototipado rápido, reproducibilidad transparente), industria (diseño preliminar sin capital en software costoso). Trabajo futuro: expansión catalizadores, GUI web, integración CFD nativa, análisis económico. Código, casos y datos disponibles en repositorio público (Declaración Disponibilidad), promoviendo transparencia y facilitando adopción/extensión.

%========================================================================
% DECLARACIÓN DE CONTRIBUCIÓN
%========================================================================
\section{Declaración de contribución de autores y colaboradores}

\noindent\textbf{Javier Salas-García}: Conceptualización, Metodología, Software, Validación, Análisis formal, Investigación, Curación de datos, Redacción original. \textbf{Miguel Moran Gonzalez}: Validación, Investigación, Redacción revisión. \textbf{María Dolores Durán García}: Metodología, Validación, Supervisión, Redacción revisión. \textbf{Rosa Romero Romero}: Recursos, Supervisión, Financiamiento, Redacción revisión. \textbf{Reyna Natividad Rangel}: Conceptualización, Recursos, Redacción revisión.

%========================================================================
% AGRADECIMIENTOS
%========================================================================
\section*{Agradecimientos}

Los autores agradecen al Centro Conjunto de Investigación en Química Sustentable UAEM–UNAM por acceso a instalaciones, al Dr. Kouzu y colaboradores por publicar datos detallados, y a revisores anónimos por sugerencias.

%========================================================================
% REFERENCIAS
%========================================================================
\vspace*{0.9\baselineskip}
\bibliographystyle{IEEEtranIDEAS.bst}
\bibliography{references}

\end{document}
