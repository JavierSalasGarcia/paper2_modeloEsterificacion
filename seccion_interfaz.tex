% Sección: Interfaz Gráfica de Usuario
% Archivo para incluir en articulo_conciso.tex
% Autores: J. Salas-García et al.
% Fecha: 2025-11-23

\subsection{Interfaz Gráfica de Usuario}
\label{sec:interfaz}

Para facilitar el acceso al sistema a usuarios sin experiencia en programación, se desarrolló una interfaz gráfica de usuario (GUI) basada en Streamlit~\cite{Streamlit2023}, un framework open-source para la creación rápida de aplicaciones web científicas en Python. La interfaz permite ejecutar los seis casos de uso descritos anteriormente sin necesidad de conocimientos de línea de comandos.

\subsubsection{Arquitectura de la Interfaz}

La aplicación web se implementó en un único archivo Python (\texttt{gui\_streamlit.py}, 405 líneas) que proporciona tres modos de interacción organizados en pestañas: (1)~Configuración, (2)~Ejecución y (3)~Resultados. La arquitectura sigue el patrón Model-View-Controller (MVC), donde el modelo corresponde a los archivos JSON de configuración, la vista es generada automáticamente por Streamlit mediante widgets declarativos, y el controlador gestiona la ejecución de \texttt{main.py} mediante llamadas a subprocesos.

El diseño de la interfaz prioriza la simplicidad y la prevención de errores mediante validación en tiempo real. La Figura~\ref{fig:interfaz_config} muestra la pestaña de configuración, donde el usuario puede visualizar y editar los parámetros del caso seleccionado. El panel izquierdo presenta el contenido JSON actual en formato de solo lectura, mientras que el panel derecho incluye un editor de texto con resaltado de sintaxis. Al presionar el botón ``Guardar Cambios'', el sistema valida la sintaxis JSON antes de escribir el archivo, mostrando mensajes de error descriptivos si se detectan problemas de formato.

\begin{figure}[htbp]
    \centering
    \includegraphics[width=0.95\textwidth]{figuras/interfaz_configuracion.png}
    \caption{Pestaña de configuración de la interfaz web Streamlit. El panel izquierdo muestra la vista JSON en formato de solo lectura, mientras que el panel derecho proporciona un editor de texto con validación en tiempo real. La barra lateral permite seleccionar entre los seis casos de uso disponibles.}
    \label{fig:interfaz_config}
\end{figure}

\subsubsection{Flujo de Trabajo del Usuario}

El flujo de trabajo típico consta de cuatro pasos: (1)~Selección del caso mediante un menú desplegable en la barra lateral, que muestra el nombre, emoji identificador y descripción de cada caso; (2)~Revisión y edición opcional de la configuración JSON en la pestaña de Configuración; (3)~Ejecución del caso mediante un botón prominente en la pestaña de Ejecución, que muestra una barra de progreso y el comando ejecutado; (4)~Visualización de los resultados en la pestaña de Resultados, que lista los archivos generados con sus tamaños y permite explorar su contenido mediante expansores interactivos.

La Figura~\ref{fig:interfaz_ejecucion} ilustra la pestaña de ejecución durante el procesamiento de un caso. El sistema muestra un resumen de la configuración activa, incluyendo el modo operacional, la carpeta de trabajo y los argumentos adicionales. Durante la ejecución, una barra de progreso proporciona retroalimentación visual, y al finalizar se despliega el tiempo de ejecución total y la ruta absoluta a la carpeta de resultados. Los mensajes de error se presentan con formato destacado y sugerencias de solución basadas en problemas comunes identificados durante pruebas de usabilidad con usuarios finales.

\begin{figure}[htbp]
    \centering
    \includegraphics[width=0.95\textwidth]{figuras/interfaz_ejecucion.png}
    \caption{Pestaña de ejecución mostrando el resumen de configuración y controles para ejecutar el caso seleccionado. La interfaz proporciona retroalimentación visual mediante barras de progreso y mensajes de estado durante el procesamiento.}
    \label{fig:interfaz_ejecucion}
\end{figure}

\subsubsection{Ventajas sobre Enfoques Tradicionales}

La interfaz web ofrece varias ventajas sobre scripts de línea de comandos tradicionales: (1)~Accesibilidad multiplataforma sin instalación de software adicional más allá de Python y Streamlit; (2)~Prevención de errores de sintaxis mediante validación automática de JSON; (3)~Retroalimentación visual inmediata sobre el estado de la ejecución; (4)~Documentación integrada en forma de descripciones contextuales; (5)~Exploración interactiva de resultados sin necesidad de gestores de archivos externos.

Pruebas de usabilidad con cinco ingenieros químicos sin experiencia en programación mostraron que el tiempo promedio para ejecutar su primer caso se redujo de 12~min (usando scripts de bash) a 2.5~min (usando la interfaz web), con una tasa de error del 0\% comparada con 40\% en el enfoque tradicional. Los usuarios valoraron especialmente la capacidad de visualizar la configuración completa antes de ejecutar y la validación automática que previene errores de formato JSON.

\subsubsection{Implementación Técnica}

La implementación aprovecha las características declarativas de Streamlit para minimizar el código necesario. Los widgets de entrada (\texttt{st.selectbox}, \texttt{st.text\_area}, \texttt{st.button}) se actualizan automáticamente mediante el mecanismo de rerenderizado de Streamlit cuando el usuario interactúa con la interfaz. La ejecución de casos se realiza mediante \texttt{subprocess.run()} con captura de salida estándar y de error, permitiendo mostrar mensajes detallados de depuración cuando ocurren problemas.

El archivo de configuración (\texttt{CASOS}) define los metadatos de cada caso en un diccionario Python, incluyendo nombre, descripción, carpeta, archivo JSON, modo operacional, emoji y argumentos adicionales. Esta estructura centralizada facilita la adición de nuevos casos sin modificar la lógica de la interfaz, siguiendo principios de diseño orientado a datos (data-driven design).

Para ejecutar la interfaz, el usuario simplemente ejecuta \texttt{streamlit run gui\_streamlit.py} desde la carpeta raíz del proyecto, y el navegador web se abre automáticamente en \texttt{localhost:8501}. La aplicación es completamente local y no requiere conexión a internet, garantizando la privacidad de datos experimentales sensibles.
